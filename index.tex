% Created 2024-03-27 Wed 09:00
% Intended LaTeX compiler: pdflatex
\documentclass[a4paper, 11pt]{article}
\usepackage[utf8]{inputenc}
\usepackage[T1]{fontenc}
\usepackage{graphicx}
\usepackage{longtable}
\usepackage{wrapfig}
\usepackage{rotating}
\usepackage[normalem]{ulem}
\usepackage{amsmath}
\usepackage{amssymb}
\usepackage{capt-of}
\usepackage{hyperref}
\usepackage[newfloat]{minted}
\RequirePackage{listings}
\RequirePackage{fancyvrb}
\DefineVerbatimEnvironment{verbatim}{Verbatim}{fontsize=\scriptsize}
\DefineVerbatimEnvironment{lstlisting}{Verbatim}{fontsize=\scriptsize}
\renewcommand\familydefault{\sfdefault}
\usepackage{python}
\usepackage{tikz}
\usepackage{amsmath, bm}
\usepackage{arev}
\usepackage{minted}
\usemintedstyle{borland}
\author{Huiyuan Chua}
\date{\today}
\title{Examination of DDPM and CEM on Bayesian Inversion problems.}
\hypersetup{
 pdfauthor={Huiyuan Chua},
 pdftitle={Examination of DDPM and CEM on Bayesian Inversion problems.},
 pdfkeywords={},
 pdfsubject={},
 pdfcreator={Emacs 29.1 (Org mode 9.6.6)}, 
 pdflang={English}}
\begin{document}

\maketitle
\tableofcontents


\section{Log}
\label{sec:org7b05726}
\subsection{2024-03-15\hfill{}\textsc{meeting}}
\label{sec:orgf9a5702}
\begin{itemize}
\item Professor Wang's server is downloading the Mayo training data.
\item For this problem, CEM has to be conditioned on three parameters (instead of two), namely:
\begin{enumerate}
\item Gaussian noise level \(t\) (applied to ground truth \(\bm{X}_0\))
\item forward diffused ground truth \(\bm{X}_t\) at noise level \(t\)
\item low dose CT image
\end{enumerate}
\item The low does CT image is a new parameter and conditions the neural network to denoise \(\bm{X}_t\) to \(\widetilde{\bm{X}_0}\) with similar probability distribution to \(\bm{X}_0\) (identical to earlier DDPM/CEM experiments/exercise).
\item The number of training samples is too little, and we will need to create more samples akin to (Adler, Jonas and Öktem, Ozan, 2018). @Huiyuan to reach out to authors on code to create new samples.
\item For this exercise, we can amend the earlier conditioned U-net to allow for two channels instead of one. Professor Wang has a sample pytorch code here \url{https://github.com/wang-zhongjian/CFNO/blob/main/codes/unet.py}.
\item Code for this problem is stored here \url{https://github.com/ainuyew/bayesian-inversion}.
\end{itemize}

\subsection{2024-03-16}
\label{sec:orgbddbe68}
\begin{itemize}
\item Found a paper and assodicated code to simulate low-dose samples at \url{https://github.com/smuzd/LD-CT-simulation/tree/master} ((Zeng, Dong and Huang, Jing and Bian, Zhaoying and Niu, Shanzhou and Zhang, Hua and Feng, Qianjin and Liang, Zhengrong and Ma, Jianhua, 2015)). The Matlab code is dependent on MIRT library (\url{https://github.com/JeffFessler/mirt}), which is not actively maintained (author has switched to Julia). The code fails to run on Octave.
\item Found another github repository (\url{https://github.com/xinario/SAGAN}; \url{https://link.springer.com/article/10.1007/s10278-018-0056-0}) working on the CT scans with similar motives (to denoise low dosage CT scans). Octave does not support all the Matlab functions (e.g. function \texttt{fanbeam} in the image package) required by this code.
\item Read a 2020 paper (\url{https://dx.doi.org/10.1088/1361-6560/ab8953}) on simulating low dose CT scans that provided a summary on advances thus far, and introduced their better models/methods.  This goes beyond earlier methods of simply introducing Poisson noise (for quantum noise) and Gaussian noise (for electronic noise). The reearch tests various models to account for negative effects such as beam hardening, and compares their models against actual low dose CT scans.
\item Alternatively ((Yi, Xin and Babyn, Paul, 2018), \url{https://arxiv.org/abs/1708.06453}), there is a 850/dose deceased piglets CT scan data set with
\end{itemize}

\subsection{2024-03-17}
\label{sec:org484154e}
\begin{itemize}
\item \url{https://github.com/odlgroup/odl/issues/1569} add Poisson noise to projection data
\item \url{https://github.com/odlgroup/odl/tree/25ec783954a85c2294ad5b76414f8c7c3cd2785d/odl/contrib/datasets/ct} code for mayo reconstruction by authors
\end{itemize}
\subsection{2024-03-18}
\label{sec:org241c149}
\begin{itemize}
\item Tried unsuccessfully to install python 3.8 to work with ODL (last released in 2018). ODL will only work with numpy < 1.20. The author has contributed the Mayo data code to ODL. Downloaded source code of ODL with the intent to correct/update the code to work with 3.10 and associated dependencies.
\end{itemize}
description as a private header tag "WaterAttenuationCoefficient" (see \url{https://www.ncbi.nlm.nih.gov/pmc/articles/PMC4644156/}). This is neccessary to calibrate the projections to water (HU=0).
\begin{itemize}
\item Unable to run sample ODL code on M1. 3D geometry ray transformation requires CUDA. Sample code show how to create training data.
\item There's matlab code  (\url{https://github.com/xinario/SAGAN/blob/master/poisson\_noise\_simulation/add\_poisson\_noise.m}) that details how it's done in reference to the paper ((Yi, Xin and Babyn, Paul, 2018)). The code adds noise to an image. The sinograms are created prior to adding of noise. A reconstructed image is returned by the function.
\end{itemize}

\begin{center}
\includegraphics[width=.9\linewidth]{20240318-192545_screenshot.png}
\end{center}

\begin{itemize}
\item (Yi, Xin and Babyn, Paul, 2018) uses images from few subjects (36 subjects). Hence, there's little independence between the images.
\end{itemize}
\subsection{2024-03-19}
\label{sec:orgea51822}
\begin{itemize}
\item National Lung Screening Trial (NLST) at \url{https://wiki.cancerimagingarchive.net/display/NLST} has 11.TB of CT scans/images for 26,254 subjects.
\item \url{https://huggingface.co/docs/diffusers/v0.25.0/en/api/models/unet2d-cond} has a conditional U-Net that ".. takes a noisy sample, conditional state, and a timestep and returns a sample shaped output.".
\item Jax conditional Unet should be able to embed multiple channels in the data i.e. it has a shape of (n\textsubscript{batch}, rows, columns, n\textsubscript{channel}) = (1000, 512, 512, 2).
\item Worked on code to train DDPM.
\end{itemize}
\subsection{2024-03-20}
\label{sec:orge5c3100}
\begin{itemize}
\item Training code is crashing with high memory consumption. Reducing number of images did not help. Will need to scale image down to see if that helps.
\item It might be worthwhile to try to run this on another library from jax in case that is the problem. There are plenty of Unet implementations on CT scans e.g. \url{https://medium.com/@fabio.sancinetti/u-net-convnet-for-ct-scan-segmentation-6cc0d465eed3}. Unet was afterall created for modeling image segmentation problems for medical applications (Ronneberger, Olaf and Fischer, Philipp and Brox, Thomas, 2015).
\end{itemize}
\subsection{2024-03-21}
\label{sec:orgbc43217}
\begin{itemize}
\item Successfully train DDPM using smaller images (128x128 instead of 512x512) with very small loss. Working on sampling code for DDPM. Will need to retrain DDPM on 90\% data to reserve remaining 10\% for testing.
\item Need to investigate on how to measure success/failure. This should largely follow the paper (Adler, Jonas and Öktem, Ozan, 2018).
\item Outstanding tasks/ideas:
\begin{itemize}
\item CEM code
\item code to create more sample data (based on ODL library) from full dose Helical sinograms?
\item the condition is based on quarter dose images. do we need different dosage?
\item run model on full resolution 512x512
\item Mayo test data has quarter dose (and omits full dose). Lesion information is available. Some qualitative checks can be performed to see if samples show the lesions in a clearer manner?
\end{itemize}
\end{itemize}
\subsection{2024-03-23}
\label{sec:orgc1baab9}
\begin{itemize}
\item managed to train DPPM and CEM on server. To update loss and numer of epochs.
\item managed to generate 1000 samples on M1 using jax-metal. The mean sample is not ideal. The problem might be the loss function used (MSE). DDPM was trained for 20 epochs on 90\% training data. The sample was based on the remaining unseen 10\% data.
\item unable to run odl sample code on local nvidia machine (ran out of memory). Will need to try this on the server.
\end{itemize}
\subsection{2024-03-26}
\label{sec:orge28b071}
\begin{itemize}
\item Reading up on batch normalization. \url{https://towardsdatascience.com/batch-norm-explained-visually-how-it-works-and-why-neural-networks-need-it-b18919692739}, \url{https://en.wikipedia.org/wiki/Batch\_normalization}, \url{https://arxiv.org/abs/1502.03167}.
\item CEM sampling (100 samples) with data rescaled to -1 to 1 seem to produce good samples. Next task at hand is to figure out if it's better than the low dose sample. We can do this by comparing how well the lesion can be seen in the contrast. See (Adler, Jonas and Öktem, Ozan, 2018).
\item Will want to consider to rescale data by z-score. mean and variance will be based on training data.
\end{itemize}
\section{Problem Statement}
\label{sec:org1b5de5b}
This is an investigation into the use of diffusion models to solve the problem discussed in (Adler, Jonas and Öktem, Ozan, 2018). Specifically, we use DDPM and CEM to model the problems. We want to investigate if DDPM and CEM will perform better than the reference method/model ((Adler, Jonas and Öktem, Ozan, 2018)) to denoise ultra low dose CT scans, and produce higher quality CT scans similar to the normal dose images.

In a CT scan of a subjectsbegy2pts, multiple 2D projections are generated for each angle of projection. Projections are created by the forward projection algorithm,and measures the attenutation (reduction in intensity) of the x-ray beam (cast no the subject towards the deector). A sinogram consist of mutiple projections stacked together. We can then reconstruct a 3D image of the subject of interest through a back projection algorithm such as filtered back projection (FBP). Filtered back projection is the industry standard to reconstruct images from sinograms because it is fast and robust. FBP applies a image de-blur filter (sharpen) to the projection (sinogram) and then back project the resulting projection to a 3D image. In back projection, we map the data in detector space to image space.



In (Adler, Jonas and Öktem, Ozan, 2018), we train two neural networks (GAN) to assist in the analysis of ultra low dose CT scans. An ultra low dose CT scan is simulated from a normal dose CT scan image. Projections are sampled from the normal dose image and Poisson noise is added to the projections. The ultra low dose CT image is then reconstructed using filtered back projection (FBP). The first neural network (deep posterior sampling) produces (high quality) sample images from the ultra low dose images. The second neural network (deep direct estimation) returns the mean and variance (of what the normal dose should be) given an ultra lose dose image.

(Moen, Taylor R. and Chen, Baiyu and Holmes III, David R. and Duan, Xinhui and Yu, Zhicong and Yu, Lifeng and Leng, Shuai and Fletcher, Joel G. and McCollough, Cynthia H., 2021) provides an overview of the data provided for the grand challenge. In particular, the low dose projection (DICOM-CT-PD) are simulated from the normal dose projections by adding Poisson noise.

\subsection{Filtered Back Projection (FBP)}
\label{sec:org58d0085}
Found a site on filtered back projection \url{https://howradiologyworks.com/filtered-backprojection-fbp-illustrated-guide-for-radiologic-technologists/\#:\~:text=Back\%20projection\%20is\%20the\%20process,Filtered\%20Backprojection\%20and\%20Iterative\%20Reconstruction}. A traditional x-ray scan gives us a 2d image of a subject of interest from a view (usually in front of the patient for a chest scan). One 2d image does not provide us a very good understanding of the subject. A forward projection algorithm instead takes multiple 2d images of a the subject from multiple and different angles/views. The end results are \uline{sinograms}. Each line/row in a sinogram corresponds to 2d image taken of the subject at particular view. A sinogram is so named because each point in the subject corresponds traces a sinusoid curve.

Given knowledge of how a forward projection works, we can reverse the process (which we call back projection) to reconstruct an image of the subject. Because the back projection (BP) is performed one view at a time, the resulting reconstructed image is a blurred image (\url{https://www.youtube.com/watch?v=YvYIkbiRMy0}). To remedy this problem, we can apply a sharpening step (or image deblurring). This image deblurring can be applied in three different manner:
\begin{enumerate}
\item BP projection to image then apply image de-blur
\item De-blur projection then BP to image
\item De-blur projection, BP to image, then finally de-blur again
\end{enumerate}
The second option (called Filtered Back Projection or FBP) is the most common as it is fast and robust. Back projection is the process of mapping the data from the detector space to the image space, while forward projection is the process of mapping the data in the image space to the detector space.

\section{Data}
\label{sec:org60d48d4}
We use data from 2016 Low Dose CT Grand Challenge (\url{https://www.aapm.org/grandchallenge/lowdosect/\#testDatasets}). Training data is downloaded from box at \url{https://aapm.app.box.com/s/eaw4jddb53keg1bptavvvd1sf4x3pe9h}. (\url{https://www.imagewisely.org/Imaging-Modalities/Computed-Tomography/Image-Reconstruction-Techniques}) The data contains images reconstructed using two reconstruction kernels B30 and D45. Reconstruction kernels (also called “filter” or “algorithm”) affects the image quality. There is a tradeoff between spatial resolution and noise. A smoother kernel generates images with lower noise but with reduced spatial resolution. A sharper kernel generates images with higher spatial resolution, but increases the image noise. Spatial resolution in CT is the ability to differentiate objects of different density. A high spatial resolution is important to distinguish objects that are close to one another.

Patient\textsubscript{Data} directory contains the reconstructed images and projections (in DICOM-CT-PD format) of the CT scans.
Ancillary\textsubscript{Information} contains detailed documentation on the file format DICOM-CT-PD (vendor neutral DICOM format), and lesion information.

Full (normal) and associate quarter (low) dose projections (.DCM) and associated reconstructed images (.IMA) are provided by Mayo clinic. The low dose projections are simulated from the normal dose projections (McCollough, Cynthia H. and Bartley, Adam C. and Carter, Rickey E. and Chen, Baiyu and Drees, Tammy A. and Edwards, Phillip and Holmes, David R. and Huang, Alice E. and Khan, Farhana and Leng, Shuai and McMillan, Kyle L. and Michalak, Gregory J. and Nunez, Kristina M. and Yu, Lifeng and Fletcher, Joel G., 2017). The training data contain data from ten patients while the testing data contain data from twenty patients. For each patient and dosage level, there are \textasciitilde{}48k projection files and 225 reconstructed images. Each 225 reconstructed images are 2D images which forms a 3D view of the patient. Mayo has provided reconstructed images using two thickness (1mm and 3mm) and two kernels (B30 and D45). Hence, for every patient, we have a total of 2 x 2 x 2 x 225 = 1800 image files.

In the Helical scan, pitch refers to the movement of the table in the z-axis relative to the height of the detector. A helical pitch of 1.0 means the table will move a distance equal to the height of the detector resulting in scans which do not overlap and do not have gaps. A helical pitch of 0.5 means the table will move a distance equal to half the height of the detector resulting in scans that overlap. This is usually done to improve spatial resolution.

We examine some full dose samples from the training images.
\begin{minted}[]{python}
from skimage.transform import iradon
from pydicom import dcmread
import matplotlib.pyplot as plt
import numpy as np
from pathlib import Path
import tqdm

import utils

path='/Users/huiyuanchua/Documents/data/Mayo_Grand_Challenge/Patient_Data'

ima_path=f'{path}/Training_Image_Data/3mm B30'
ima_fd_path=f'{ima_path}/full_3mm/L067/full_3mm'

pathlist = Path(ima_fd_path).rglob('*.IMA')
ima_files = [ima for ima in pathlist]

n = 3
images=[]
ima_batch = np.random.choice(ima_files, size=n**2, replace=False)
fig, axs = plt.subplots(n, n, figsize=(3 *n, 3 * n), sharex=True, sharey=True)
_ = fig.tight_layout()
for i, ima_file in tqdm.tqdm(enumerate(ima_batch)):
    ima = dcmread(ima_file)
    image = ima.pixel_array

    # convert to HU
    hu_values = ima.RescaleSlope * image + ima.RescaleIntercept
    densities = (hu_values + 1000)/1000
    images.append(densities)
    ax = axs[i // n][i % n]
    #ax.imshow(image, cmap=plt.cm.Greys_r,)
    ax.imshow(utils.normalize_image(densities), cmap=plt.cm.Greys_r,)
plt.show()
\end{minted}

\begin{verbatim}
9it [00:00, 282.17it/s]

\end{verbatim}

\begin{center}
\includegraphics[width=.9\linewidth]{./.ob-jupyter/d106bffcb7948cba61b5dba49266665e75137c41.png}
\end{center}


We apply forward projection on the batch to obtain their respective sinograms (Radon transform).
\begin{minted}[,frame=single, linenos, breaklines, tabsize=2]{python}
import numpy as np
import matplotlib.pyplot as plt
from skimage.data import shepp_logan_phantom
from skimage.transform import radon, rescale
import tqdm

sinograms = []
fig, axs = plt.subplots(n, n, figsize=(3 *n, 3 * n), sharex=True, sharey=True)
_ = fig.tight_layout()
for i, image in tqdm.tqdm(enumerate(images)):
    ax = axs[i // n][i % n]

    theta = np.linspace(0., 180., max(image.shape), endpoint=False)
    sinogram = radon(image, theta=theta)
    sinograms.append(sinogram)
    dx, dy = 0.5 * 180.0 / max(image.shape), 0.5 / sinogram.shape[0]
    ax.imshow(sinogram, cmap=plt.cm.Greys_r,
               extent=(-dx, 180.0 + dx, -dy, sinogram.shape[0] + dy),
               aspect='auto')

    ax.imshow(sinogram)
plt.show()
\end{minted}

\begin{verbatim}
0it [00:00, ?it/s]/Users/huiyuanchua/miniconda3/envs/venv310/lib/python3.10/site-packages/skimage/transform/radon_transform.py:75: UserWarning: Radon transform: image must be zero outside the reconstruction circle
  warn('Radon transform: image must be zero outside the '
9it [00:06,  1.42it/s]

\end{verbatim}

\begin{center}
\includegraphics[width=.9\linewidth]{./.ob-jupyter/9908590d1629c61ffc91bfa8d539c672a7a6b737.png}
\end{center}

We now apply filtered back projection to reconstruct the original image using the filter "ramp".
\begin{minted}[,frame=single, linenos, breaklines, tabsize=2]{python}
from skimage.transform import iradon
from pydicom import dcmread
import matplotlib.pyplot as plt

import utils

sinogram = sinograms[4]
image = images[4]
reconstruction_fbp = iradon(sinogram, theta=theta, filter_name='ramp')
error = reconstruction_fbp - image
print(f'FBP rms reconstruction error: {np.sqrt(np.mean(error**2)):.3g}')

imkwargs = dict(vmin=-0.2, vmax=0.2)
fig, (ax1, ax2, ax3) = plt.subplots(1, 3, figsize=(8, 4.5), sharex=True, sharey=True)
_ = fig.tight_layout()
ax1.set_title("Ground Truth")
ax1.imshow(utils.normalize_image(image), cmap=plt.cm.Greys_r)
ax2.set_title("FBP image")
ax2.imshow(utils.normalize_image(reconstruction_fbp), cmap=plt.cm.Greys_r)
ax3.set_title("Reconstruction error\nFiltered back projection")
ax3.imshow(error, cmap=plt.cm.Greys_r, **imkwargs)
plt.show()
\end{minted}

\begin{verbatim}
FBP rms reconstruction error: 0.105
\end{verbatim}

\begin{center}
\includegraphics[width=.9\linewidth]{./.ob-jupyter/aae84632d2a803a7aac0bf7c4eb669df0b4008ec.png}
\end{center}

Here, we retrieve Mayo training data in pairs (full dose, quarter dose), and compare a few samples.
\begin{minted}[,frame=single, linenos, breaklines, tabsize=2]{python}
import mayo
import numpy as np
import matplotlib.pyplot as plt

import utils

path='/Users/huiyuanchua/Documents/data/Mayo_Grand_Challenge/Patient_Data/Training_Image_Data/3mm B30'
training_data = mayo.get_training_data(path, 112, 113) # pull the middle slices of each patient
fd_ima, qd_ima = training_data[len(training_data) // 2]
noise = fd_ima - qd_ima

print(f'noise: {np.sqrt(np.mean(noise**2)):.3g}')

imkwargs = dict(vmin=-0.2, vmax=0.2)
fig, (ax1, ax2, ax3) = plt.subplots(1, 3, figsize=(8, 4.5), sharex=True, sharey=True)
_ = fig.tight_layout()
_ = ax1.set_title("full dose image")
_ = ax1.imshow(utils.normalize_image(fd_ima), cmap=plt.cm.Greys_r)
_ = ax2.set_title("quarter dose image")
_ = ax2.imshow(utils.normalize_image(qd_ima), cmap=plt.cm.Greys_r)
_ = ax3.set_title("noise")
_ = ax3.imshow(noise, cmap=plt.cm.Greys_r, **imkwargs)
plt.show()

\end{minted}

\begin{verbatim}
loading patient data: 100% 10/10 [00:00<00:00, 65.98it/s]

noise: 0.00468
\end{verbatim}

\begin{center}
\includegraphics[width=.9\linewidth]{./.ob-jupyter/87b4f042c9bbc291a64110740ae09645af97d617.png}
\end{center}

Next, we explore if we can reconstruct images from Mayo's full dose projections (Helical scans), and how it compares with the provided images reconstructed using commercial software (weighted FBP).
\begin{minted}[,frame=single, linenos, breaklines, tabsize=2]{python}
#from pydicom import dcmread
#import matplotlib.pyplot as plt
#import numpy as np
#from pathlib import Path

path='/home/huiyuanchua/Documents/data/Mayo_Grand_Challenge/Patient_Data'

dcm_path=f'{path}/Training_Projection_Data/L067'
dcm_fd_path=f'{dcm_path}/DICOM-CT-PD_FD'

import odl
from odl.contrib.datasets.ct import mayo

mayo_dir = '/home/huiyuanchua/Documents/data/Mayo_Grand_Challenge/Patient_Data'  # replace with your local folder

# Load reference reconstruction
#volume_folder = mayo_dir + '/Training Cases/L067/full_1mm_sharp'
volume_folder = mayo_dir + '/Training_Image_Data/3mm B30/full_3mm/L067/full_3mm'
partition, volume = mayo.load_reconstruction(volume_folder)

# Load a subset of the projection data
#data_folder = mayo_dir + '/Training Cases/L067/full_DICOM-CT-PD'
data_folder = mayo_dir + '/Training_Projection_Data/L067/DICOM-CT-PD_FD'
geometry, proj_data = mayo.load_projections(data_folder,
                                          indices=slice(20000, 28000))

# Reconstruction space and ray transform
space = odl.uniform_discr_frompartition(partition, dtype='float32')
ray_trafo = odl.tomo.RayTransform(space, geometry)

# Define FBP operator
fbp = odl.tomo.fbp_op(ray_trafo, padding=True)

# Tam-Danielsson window to handle redundant data
td_window = odl.tomo.tam_danielson_window(ray_trafo, n_pi=3)

# Calculate FBP reconstruction
fbp_result = fbp(td_window * proj_data)

# Compare the computed recon to reference reconstruction (coronal slice)
ref = space.element(volume)
fbp_result.show('Recon (coronal)', clim=[0.7, 1.3])
ref.show('Reference (coronal)', clim=[0.7, 1.3])
(ref - fbp_result).show('Diff (coronal)', clim=[-0.1, 0.1])

# Also visualize sagittal slice (note that we only used a subset)
coords = [0, None, None]
fbp_result.show('Recon (sagittal)', clim=[0.7, 1.3], coords=coords)
ref.show('Reference (sagittal)', clim=[0.7, 1.3], coords=coords)
(ref - fbp_result).show('Diff (sagittal)', clim=[-0.1, 0.1], coords=coords)

#n = 3
#sinograms=[]
#sinogram_batch = np.random.choice(dcm_files, size=n**2, replace=False)
#fig, axs = plt.subplots(n, n, figsize=(3 *n, 3 * n), sharex=True, sharey=True)
#_ = fig.tight_layout()
#for i, sinogram_file in enumerate(sinogram_batch):
#    dcm = dcmread(sinogram_file)
#    sinogram = dcm.pixel_array
#    sinograms.append(sinogram)
#    ax = axs[i // n][i % n]
#    ax.imshow(sinogram, cmap=plt.cm.Greys_r,)
#plt.show()
\end{minted}

\begin{verbatim}

ImportErrorTraceback (most recent call last)
~/miniconda3/envs/odlenv/lib/python3.6/site-packages/dask/array/chunk.py in <module>
     14 try:
---> 15     from numpy import take_along_axis
     16 except ImportError:  # pragma: no cover

ImportError: cannot import name 'take_along_axis'

During handling of the above exception, another exception occurred:

AttributeErrorTraceback (most recent call last)
<ipython-input-1-c46f9b485a61> in <module>
      9 dcm_fd_path=f'{dcm_path}/DICOM-CT-PD_FD'
     10 
---> 11 import odl
     12 from odl.contrib.datasets.ct import mayo
     13 

~/miniconda3/envs/odlenv/lib/python3.6/site-packages/odl/__init__.py in <module>
     64 from . import phantom
     65 from . import solvers
---> 66 from . import tomo
     67 from . import trafos
     68 from . import ufunc_ops

~/miniconda3/envs/odlenv/lib/python3.6/site-packages/odl/tomo/__init__.py in <module>
     17 __all__ += geometry.__all__
     18 
---> 19 from .backends import *
     20 __all__ += backends.__all__
     21 

~/miniconda3/envs/odlenv/lib/python3.6/site-packages/odl/tomo/backends/__init__.py in <module>
     22 __all__ += astra_cuda.__all__
     23 
---> 24 from .skimage_radon import *
     25 __all__ += skimage_radon.__all__

~/miniconda3/envs/odlenv/lib/python3.6/site-packages/odl/tomo/backends/skimage_radon.py in <module>
     12 import numpy as np
     13 try:
---> 14     import skimage
     15     SKIMAGE_AVAILABLE = True
     16 except ImportError:

~/miniconda3/envs/odlenv/lib/python3.6/site-packages/skimage/__init__.py in <module>
    125 
    126     # All skimage root imports go here
--> 127     from .util.dtype import (img_as_float32,
    128                              img_as_float64,
    129                              img_as_float,

~/miniconda3/envs/odlenv/lib/python3.6/site-packages/skimage/util/__init__.py in <module>
      4 from .shape import view_as_blocks, view_as_windows
      5 from .noise import random_noise
----> 6 from .apply_parallel import apply_parallel
      7 
      8 from .arraycrop import crop

~/miniconda3/envs/odlenv/lib/python3.6/site-packages/skimage/util/apply_parallel.py in <module>
      6 
      7 try:
----> 8     import dask.array as da
      9     dask_available = True
     10 except ImportError:

~/miniconda3/envs/odlenv/lib/python3.6/site-packages/dask/array/__init__.py in <module>
      1 try:
----> 2     from .blockwise import blockwise, atop
      3     from .core import (
      4         Array,
      5         block,

~/miniconda3/envs/odlenv/lib/python3.6/site-packages/dask/array/blockwise.py in <module>
    286 
    287 
--> 288 from .core import new_da_object

~/miniconda3/envs/odlenv/lib/python3.6/site-packages/dask/array/core.py in <module>
     20 import numpy as np
     21 
---> 22 from . import chunk
     23 from .. import config, compute
     24 from ..base import (

~/miniconda3/envs/odlenv/lib/python3.6/site-packages/dask/array/chunk.py in <module>
     15     from numpy import take_along_axis
     16 except ImportError:  # pragma: no cover
---> 17     take_along_axis = npcompat.take_along_axis
     18 
     19 

AttributeError: module 'dask.array.numpy_compat' has no attribute 'take_along_axis'
\end{verbatim}

\section{DDPM}
\label{sec:org249e0d8}
DDPM specific parameters.
\begin{minted}[,frame=single, linenos, breaklines, tabsize=2]{python}
import os

SEED=42
MIN_BETA, MAX_BETA = 1e-4, 0.02
K = 200
N_EPOCH = 30
BATCH_SIZE = 10
PROJECT_DIR=os.path.abspath('.')
\end{minted}

\subsection{Training}
\label{sec:org2105e76}
First, let's make sure we have the correct shapes for training data inputs and predictions.
\begin{minted}[,frame=single, linenos, breaklines, tabsize=2]{python}
from jax import random
import jax.numpy as jnp
import numpy as np

import utils
import mayo

key = random.PRNGKey(42)
state = utils.create_training_state(key=key)
path='/Users/huiyuanchua/Documents/data/Mayo_Grand_Challenge/Patient_Data/Training_Image_Data/3mm B30'
training_data = mayo.get_training_data(path, 112, 113)

x_0_fd = training_data[:, 0]
x_0_ld = training_data[:, 1]
x_k = jnp.concatenate((x_0_fd, x_0_ld), axis=-1)
n = training_data.shape[0]
k = random.choice(key, np.arange(200), shape=(n,))
predictions = state.apply_fn(state.params, x_k, k)
print(f'training input shape (x_t, t): {x_k.shape}, {k.shape}')
print(f'predictions shape (x_0): {predictions.shape}')
\end{minted}

\begin{verbatim}
loading patient data: 100% 10/10 [00:00<00:00, 76.08it/s]

training input shape (x_t, t): (10, 128, 128, 2) {(10,)}
predictions shape (x_0): (10, 128, 128, 1)
\end{verbatim}

Training is performed via a python script \url{train\_ddpm.py}. We examine the average epoch loss with the following code:
\begin{minted}[,frame=single, linenos, breaklines, tabsize=2]{python}
import pandas as pd
import matplotlib.pyplot as plt
import seaborn as sns
import os

import utils

PROJECT_DIR=os.path.abspath('.')
ddpm_loss_log = utils.load_loss_log(f'{PROJECT_DIR}/ddpm_loss_log.npy')

# plot losses
df = pd.DataFrame([(int(x), float(y)) for x, _, y in ddpm_loss_log], columns=['epoch', 'loss'])
sns.relplot(df, x='epoch', y='loss', kind='line')

_ = plt.tight_layout()
_ = plt.show()
\end{minted}

\begin{center}
\includegraphics[width=.9\linewidth]{./.ob-jupyter/98b946f3ee6dbe5becc443cc009687eff61363d4.png}
\end{center}

\subsection{Sampling}
\label{sec:org132098a}
We sample images from the trained DDPM model.
\begin{minted}[,frame=single, linenos, breaklines, tabsize=2]{python}
import matplotlib.pyplot as plt
import optax
from jax import random
import jax.numpy as jnp
from tqdm import tqdm
import numpy as np

import utils
import mayo

def sample(state, condition, n, betas, key):
  # random white noise X_T
  key, subkey = random.split(key)
  x_k = random.normal(subkey, shape=(n, condition.shape[0], condition.shape[1], 1))

  #dts = np.array([ts[i] - ts[i-1] for i in range(1, steps+1)])
  #betas = 1- np.exp(-dts)
  alphas = 1 - betas
  alpha_bars = jnp.cumprod(alphas)
  #alpha_bars = jnp.array([alphas[:i+1].prod() for i in range(len(alphas))]) # workaround for metal problem with jnp.cumprod

  condition = np.repeat(condition.reshape((1, condition.shape[0], condition.shape[1], 1)), n, axis=0)

  # sample in reverse from T=10 to 0.0 in evenly distributed steps
  #for i in tqdm(range(steps)[::-1]):
  for k in range(len(betas))[::-1]:
    alpha = alphas[k]
    beta = betas[k]
    alpha_bar_k = alpha_bars[k]

    key, subkey = random.split(key)
    z = jnp.where(k > 1, random.normal(subkey, shape=x_k.shape), jnp.zeros_like(x_k))
    sigma_k = jnp.sqrt(beta) # option 1; see DDPM 3.2
    #sigma_k = jnp.sqrt((1-alpha_bars[k-1])/(1 - alpha_bar_k) * beta) # option 2; see DDPM 3.2

    inputs = jnp.concatenate((x_k, condition), axis=-1)

    x_k = 1/jnp.sqrt(alpha) * (x_k - beta/jnp.sqrt(1 - alpha_bar_k) * state.apply_fn(state.params, inputs, k * jnp.ones((x_k.shape[0], )))) + sigma_k * z

    #x_k = jnp.clip(x_k, -1., 1.) # should we clip ...
    x_t = jnp.clip(x_t, -3., 3.)
    #x_t = normalize_to_neg_one_to_one(x_t) # or scale?

  return x_k

key = random.PRNGKey(SEED)

# use the best params
file_path, epoch, step, loss = utils.find_latest_pytree(f'{PROJECT_DIR}/ddpm_params_*.npy')
ddpm_state = utils.create_training_state(params_file=f'{PROJECT_DIR}/ddpm_params_{epoch}_{step}_{loss}.npy')
print(f'using parameters from epoch {epoch} with loss {loss}')

betas = jnp.linspace(MIN_BETA, MAX_BETA, K)

path='/Users/huiyuanchua/Documents/data/Mayo_Grand_Challenge/Patient_Data/Training_Image_Data/3mm B30'
training_data = mayo.get_training_data(path, 112, 113)
n = (len(training_data) // 10) * 9
test_data = training_data[n:]
test_data = []
n_samples = 1000

fd_data, ld_data = training_data[0]
result = np.zeros(fd_data.shape)

for i in tqdm(range(n_samples // BATCH_SIZE)):
  # generate x_0 from noise
  key, subkey = random.split(key)
  x_0_tilde = sample(ddpm_state, ld_data, BATCH_SIZE, betas, subkey)

  result = result + np.sum(x_0_tilde, axis=0)

result = result / n_samples
error = fd_data - result
print(f'error: {np.mean(error)}')

# plot and compare last sample
imkwargs = dict(vmin=-0.2, vmax=0.2)
fig, axs = plt.subplots(2, 2, figsize=(8, 4.5), sharex=True, sharey=True)
_ = fig.tight_layout()
_ = axs[0, 0].set_title("full dose image")
_ = axs[0, 0].imshow(utils.normalize_image_to_greyscale(fd_data), cmap=plt.cm.Greys_r)
_ = axs[0, 1].set_title("quarter dose image")
_ = axs[0, 1].imshow(utils.normalize_image_to_greyscale(ld_data), cmap=plt.cm.Greys_r)
_ = axs[1, 0].set_title("DDPM")
_ = axs[1, 0].imshow(utils.normalize_image_to_greyscale(result), cmap=plt.cm.Greys_r)
_ = axs[1, 1].set_title("error")
_ = axs[1, 1].imshow(error, cmap=plt.cm.Greys_r, **imkwargs)

plt.show()
\end{minted}

\begin{verbatim}
  Platform 'METAL' is experimental and not all JAX functionality may be correctly supported!
  2024-03-23 12:48:02.264962: W pjrt_plugin/src/mps_client.cc:563] WARNING: JAX Apple GPU support is experimental and not all JAX functionality is correctly supported!
  Metal device set to: Apple M1
  using parameters from epoch 22 with loss 0.00825
  loading patient data: 100% 10/10 [00:00<00:00, 59.47it/s]

  100% 200/200 [03:55<00:00,  1.18s/it]

  100% 200/200 [03:50<00:00,  1.15s/it]
100% 200/200 [03:49<00:00,  1.15s/it]
100% 200/200 [03:48<00:00,  1.14s/it]
100% 200/200 [03:48<00:00,  1.14s/it]
100% 200/200 [03:48<00:00,  1.14s/it]
100% 200/200 [03:50<00:00,  1.15s/it]
100% 200/200 [03:49<00:00,  1.15s/it]
100% 200/200 [03:49<00:00,  1.15s/it]
100% 200/200 [03:49<00:00,  1.15s/it]
100% 200/200 [03:48<00:00,  1.14s/it]
100% 200/200 [03:48<00:00,  1.14s/it]
100% 200/200 [03:48<00:00,  1.14s/it]
100% 200/200 [03:48<00:00,  1.14s/it]
100% 200/200 [03:48<00:00,  1.14s/it]
100% 200/200 [03:48<00:00,  1.14s/it]
100% 200/200 [03:54<00:00,  1.17s/it]
100% 200/200 [03:48<00:00,  1.14s/it]
100% 200/200 [03:48<00:00,  1.14s/it]
100% 200/200 [03:48<00:00,  1.14s/it]
100% 200/200 [03:49<00:00,  1.15s/it]
100% 200/200 [03:48<00:00,  1.14s/it]
100% 200/200 [03:48<00:00,  1.14s/it]
100% 200/200 [03:48<00:00,  1.14s/it]
100% 200/200 [03:51<00:00,  1.16s/it]
100% 200/200 [03:48<00:00,  1.14s/it]
100% 200/200 [03:49<00:00,  1.15s/it]
100% 200/200 [03:48<00:00,  1.14s/it]
100% 200/200 [03:48<00:00,  1.14s/it]
100% 200/200 [03:52<00:00,  1.16s/it]
100% 200/200 [03:50<00:00,  1.15s/it]
100% 200/200 [03:52<00:00,  1.16s/it]
100% 200/200 [03:51<00:00,  1.16s/it]
100% 200/200 [03:49<00:00,  1.15s/it]
100% 200/200 [03:49<00:00,  1.15s/it]
100% 200/200 [03:49<00:00,  1.15s/it]
100% 200/200 [03:49<00:00,  1.15s/it]
100% 200/200 [03:48<00:00,  1.14s/it]
100% 200/200 [03:49<00:00,  1.15s/it]
100% 200/200 [03:51<00:00,  1.16s/it]
100% 200/200 [03:49<00:00,  1.15s/it]
100% 200/200 [03:48<00:00,  1.14s/it]
100% 200/200 [03:49<00:00,  1.15s/it]
100% 200/200 [03:48<00:00,  1.14s/it]
100% 200/200 [03:49<00:00,  1.15s/it]
100% 200/200 [03:49<00:00,  1.15s/it]
100% 200/200 [03:49<00:00,  1.15s/it]
100% 200/200 [03:52<00:00,  1.16s/it]
100% 200/200 [03:49<00:00,  1.15s/it]
100% 200/200 [03:49<00:00,  1.15s/it]
100% 200/200 [03:49<00:00,  1.15s/it]
100% 200/200 [03:49<00:00,  1.15s/it]
100% 200/200 [03:49<00:00,  1.15s/it]
100% 200/200 [03:49<00:00,  1.15s/it]
100% 200/200 [03:49<00:00,  1.15s/it]
100% 200/200 [03:52<00:00,  1.16s/it]
100% 200/200 [03:49<00:00,  1.15s/it]
100% 200/200 [03:49<00:00,  1.15s/it]
100% 200/200 [03:49<00:00,  1.15s/it]
100% 200/200 [03:49<00:00,  1.15s/it]
100% 200/200 [03:49<00:00,  1.15s/it]
100% 200/200 [03:49<00:00,  1.15s/it]
100% 200/200 [03:50<00:00,  1.15s/it]
100% 200/200 [03:53<00:00,  1.17s/it]
100% 200/200 [03:49<00:00,  1.15s/it]
100% 200/200 [03:49<00:00,  1.15s/it]
100% 200/200 [03:49<00:00,  1.15s/it]
100% 200/200 [03:49<00:00,  1.15s/it]
100% 200/200 [03:49<00:00,  1.15s/it]
100% 200/200 [03:49<00:00,  1.15s/it]
100% 200/200 [03:49<00:00,  1.15s/it]
100% 200/200 [03:52<00:00,  1.16s/it]
100% 200/200 [03:49<00:00,  1.15s/it]
100% 200/200 [03:49<00:00,  1.15s/it]
100% 200/200 [03:49<00:00,  1.15s/it]
100% 200/200 [03:49<00:00,  1.15s/it]
100% 200/200 [03:49<00:00,  1.15s/it]
100% 200/200 [03:49<00:00,  1.15s/it]
100% 200/200 [03:52<00:00,  1.16s/it]
100% 200/200 [03:50<00:00,  1.15s/it]
100% 200/200 [03:49<00:00,  1.15s/it]
100% 200/200 [03:50<00:00,  1.15s/it]
100% 200/200 [03:49<00:00,  1.15s/it]
100% 200/200 [03:49<00:00,  1.15s/it]
100% 200/200 [03:49<00:00,  1.15s/it]
100% 200/200 [03:50<00:00,  1.15s/it]
100% 200/200 [03:52<00:00,  1.16s/it]
100% 200/200 [03:49<00:00,  1.15s/it]
100% 200/200 [03:49<00:00,  1.15s/it]
100% 200/200 [03:49<00:00,  1.15s/it]
100% 200/200 [03:49<00:00,  1.15s/it]
100% 200/200 [03:49<00:00,  1.15s/it]
100% 200/200 [03:49<00:00,  1.15s/it]
100% 200/200 [03:50<00:00,  1.15s/it]
100% 200/200 [03:52<00:00,  1.16s/it]
100% 200/200 [03:50<00:00,  1.15s/it]
100% 200/200 [03:50<00:00,  1.15s/it]
100% 200/200 [03:50<00:00,  1.15s/it]
100% 200/200 [03:50<00:00,  1.15s/it]
100% 200/200 [03:50<00:00,  1.15s/it]
  error: 0.11948700249195099
\end{verbatim}
\begin{center}
\includegraphics[width=.9\linewidth]{./.ob-jupyter/da8b230284e5503497e7e30a76f4e73bb3cc5842.png}
\end{center}

\section{CEM}
\label{sec:orge7a722e}
\begin{minted}[,frame=single, linenos, breaklines, tabsize=2]{python}
import os

SEED=42
T=10.
K=1000
BATCH_SIZE = 10
PROJECT_DIR=os.path.abspath('.')
\end{minted}
\subsection{Training}
\label{sec:orgc797b54}
Training is performed via a python script \url{train\_cem.py}. We examine the average epoch loss with the following code:
\begin{minted}[,frame=single, linenos, breaklines, tabsize=2]{python}
import pandas as pd
import matplotlib.pyplot as plt
import seaborn as sns

import utils

cem_loss_log = utils.load_loss_log(f'{PROJECT_DIR}/cem_loss_log.npy')

# plot losses
df = pd.DataFrame([(int(x), float(y)) for x, _, y in cem_loss_log], columns=['epoch', 'loss'])
sns.relplot(df, x='epoch', y='loss', kind='line')

_ = plt.tight_layout()
_ = plt.show()
\end{minted}

\begin{center}
\includegraphics[width=.9\linewidth]{./.ob-jupyter/08881e0d9cccd48adbf93a280496bacfad22854b.png}
\end{center}

\subsection{Sampling}
\label{sec:org10bb790}
We sample images from the trained CEM model.
\begin{minted}[,frame=single, linenos, breaklines, tabsize=2]{python}
import matplotlib.pyplot as plt
import optax
from jax import random
import jax.numpy as jnp
from tqdm import tqdm
import numpy as np

import utils
from unet import Unet
import mayo

def sample(state, condition, n, ts, key):
  # random white noise X_T
  key, subkey = random.split(key)
  x_t = random.normal(subkey, shape=(n, condition.shape[0], condition.shape[1], 1))

  condition = np.repeat(condition.reshape((1, condition.shape[0], condition.shape[1], 1)), n, axis=0)

  step=0

  for k in range(len(ts))[::-1]:
    key, subkey = random.split(key)
    z = random.normal(subkey, shape=x_t.shape)

    t = ts[k]
    dt = jnp.where(k > 0, t - ts[k-1], 0.)

    inputs = jnp.concatenate((x_t, condition), axis=-1)

    f_theta = state.apply_fn(state.params, inputs, t * jnp.ones((n,)))

    # equation (40)
    s_theta = jnp.where(k > 0, x_t/(1-jnp.exp(-t))  - jnp.exp(-t/2)/(1-jnp.exp(-t)) * f_theta,  0.)

    # equation (24)
    x_t_bar = x_t - dt * s_theta
    x_t = jnp.exp(dt/2) * x_t_bar + jnp.sqrt(1-jnp.exp(-dt)) * z

    #x_t = jnp.clip(x_t, -1., 1.) # should we clip ...
    x_t = jnp.clip(x_t, -3., 3.)
    #x_t = normalize_to_neg_one_to_one(x_t) # or scale?

    step=step+1

  return x_t

key = random.PRNGKey(SEED)
n_samples = 100

# use the best params
file_path, epoch, step, loss = utils.find_latest_pytree(f'{PROJECT_DIR}/cem_params_*.npy')
cem_state = utils.create_training_state(params_file=file_path)
print(f'using parameters from epoch {epoch} with loss {loss}')

ts = utils.exponential_time_schedule(T, K)
#path='/home/gpu_user1/Documents/data/Mayo_Grand_Challenge/Patient_Data/Training_Image_Data/3mm B30'
path='/Users/huiyuanchua/Documents/data/Mayo_Grand_Challenge/Patient_Data/Training_Image_Data/3mm B30'
training_data = mayo.get_training_data(path, 112, 113)
n = (len(training_data) // 10) * 9
test_data = training_data[n:]

fd_data, ld_data = training_data[0]
result = np.zeros(fd_data.shape)

for i in tqdm(range(n_samples // BATCH_SIZE)):
  # generate x_0 from noise
  key, subkey = random.split(key)
  x_0_tilde = sample(cem_state, ld_data, BATCH_SIZE, ts, subkey)

  result = result + np.sum(x_0_tilde, axis=0)

result = result / n_samples
error = fd_data - result
print(f'error: {np.mean(error)}')

# plot the data
imkwargs = dict(vmin=-0.2, vmax=0.2)
fig, axs = plt.subplots(2, 2, figsize=(8, 4.5), sharex=True, sharey=True)
_ = fig.tight_layout()
_ = axs[0, 0].set_title("full dose image")
_ = axs[0, 0].imshow(utils.normalize_image_to_greyscale(fd_data), cmap=plt.cm.Greys_r)
_ = axs[0, 1].set_title("quarter dose image")
_ = axs[0, 1].imshow(utils.normalize_image_to_greyscale(ld_data), cmap=plt.cm.Greys_r)
_ = axs[1, 0].set_title("CEM")
_ = axs[1, 0].imshow(utils.normalize_image_to_greyscale(result), cmap=plt.cm.Greys_r)
_ = axs[1, 1].set_title("error")
_ = axs[1, 1].imshow(error, cmap=plt.cm.Greys_r, **imkwargs)

plt.show()
\end{minted}

\begin{verbatim}
using parameters from epoch 73 with loss 6e-05
loading patient data: 100% 10/10 [00:00<00:00, 60.13it/s]

100% 10/10 [3:15:07<00:00, 1170.71s/it]

error: -0.015812324360013008
\end{verbatim}

\begin{center}
\includegraphics[width=.9\linewidth]{./.ob-jupyter/2fc823732f5bee1ceac65ab637b5daee3bbfc03a.png}
\end{center}
:END:

\section{Reads}
\label{sec:orgfea6ca8}
\begin{itemize}
\item Quick summary on CR reconstruciton and Helical CT \url{http://xrayphysics.com/ctsim.html}
\item Example to perform simple image reconstruction using scikit-image \url{https://scikit-image.org/docs/stable/auto\_examples/transform/plot\_radon\_transform.html}
\item Python code to reconstruct image from helical scans? \url{https://github.com/dzwiedzn7/filtered-back-projection/blob/master/tompy.py}
\item Python code to simulate thick slice images from Helical scans \url{https://github.com/Feanor007/Thin2Thick}
\item Python library for Tomography \url{https://pypi.org/project/algotom/}
\item C code for Model-Based Iterative Reconstruciton code for Multi-Slice Helical Geometry \url{https://github.com/cabouman/OpenMBIR-Index/blob/master/README.md}
\item General summary for 3D image reconstruction \url{https://humanhealth.iaea.org/HHW/MedicalPhysics/NuclearMedicine/ImageAnalysis/3Dimagereconstruction/index.html}
\item Pyro-NN: Generalized Python code for image reconstruction using deep learning implemented in Tensorflow code \url{https://www.ncbi.nlm.nih.gov/pmc/articles/PMC6899669/}
\item TomoPy: python library for tomographic data analysis \url{https://www.ncbi.nlm.nih.gov/pmc/articles/PMC4181643/}
\item Powerpoint presentation on CT image reconstruction \url{http://www.sci.utah.edu/\~shireen/pdfs/tutorials/Elhabian\_CT09.pdf}
\item Astra toolbox: python matlab library for 2D/3D tomography \url{https://github.com/astra-toolbox/astra-toolbox}
\item Operator Discretization Library (ODL): used by authors of the paper and has example Python code to reconstruct image from Helical scans \url{https://github.com/odlgroup/odl/tree/master/examples/tomo}
\item matlab code for simple low-dose CT samples simulation \url{https://github.com/smuzd/LD-CT-simulation/tree/master}
\item matlab code for 2nd winner from 2016 Mayo Grand Challenge \url{https://github.com/jongcye/deeplearningLDCT/tree/master}
\item Jupyter code to add white noise to CT scans. May have code to reconstruct images from Helical scans? \url{https://github.com/ayaanzhaque/Noise2Quality}
\item Alternative CT training data? \url{https://www.kaggle.com/c/data-science-bowl-2017/overview}
\item Alternative CT training data from piglets? \url{https://link.springer.com/article/10.1007/s10278-018-0056-0}
\end{itemize}
\section{References}
\label{sec:org22f4503}
\noindent
Adler, Jonas and Öktem, Ozan (2018). \emph{Deep {{Bayesian Inversion}}}.

\noindent
McCollough, Cynthia H. and Bartley, Adam C. and Carter, Rickey E. and Chen, Baiyu and Drees, Tammy A. and Edwards, Phillip and Holmes, David R. and Huang, Alice E. and Khan, Farhana and Leng, Shuai and McMillan, Kyle L. and Michalak, Gregory J. and Nunez, Kristina M. and Yu, Lifeng and Fletcher, Joel G. (2017). \emph{Low-Dose {{CT}} for the Detection and Classification of Metastatic Liver Lesions: {{Results}} of the 2016 {{Low Dose CT Grand Challenge}}}.

\noindent
Moen, Taylor R. and Chen, Baiyu and Holmes III, David R. and Duan, Xinhui and Yu, Zhicong and Yu, Lifeng and Leng, Shuai and Fletcher, Joel G. and McCollough, Cynthia H. (2021). \emph{Low-Dose {{CT}} Image and Projection Dataset}.

\noindent
Ronneberger, Olaf and Fischer, Philipp and Brox, Thomas (2015). \emph{U-{{Net}}: {{Convolutional Networks}} for {{Biomedical Image Segmentation}}}.

\noindent
Yi, Xin and Babyn, Paul (2018). \emph{Sharpness-Aware {{Low}} Dose {{CT}} Denoising Using Conditional Generative Adversarial Network}.

\noindent
Zeng, Dong and Huang, Jing and Bian, Zhaoying and Niu, Shanzhou and Zhang, Hua and Feng, Qianjin and Liang, Zhengrong and Ma, Jianhua (2015). \emph{A {{Simple Low-dose X-ray CT Simulation}} from {{High-dose Scan}}}.
\end{document}